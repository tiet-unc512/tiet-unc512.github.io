% Created 2026-02-10 Tue 12:20
% Intended LaTeX compiler: pdflatex
\documentclass[11pt]{article}
\usepackage[utf8x]{inputenc}
\usepackage[T1]{fontenc}
\usepackage{graphicx}
\usepackage{longtable}
\usepackage{wrapfig}
\usepackage{rotating}
\usepackage[normalem]{ulem}
\usepackage{amsmath}
\usepackage{amssymb}
\usepackage{capt-of}
\usepackage{hyperref}
\usepackage{minted}
\usepackage{parskip}
\graphicspath{ {./images/} }
\usepackage[labelformat=simple]{subcaption}
\renewcommand\thesubfigure{(\alph{subfigure})}
\usepackage[date=year,%
backend=biber,%
style=alphabetic,%
maxnames=5,%
minnames=3,%
maxalphanames=4,%
minalphanames=3,%
backref=true,%
doi=false,%
isbn=false,%
url=false,%
eprint=false]{biblatex}
\DefineBibliographyStrings{english}{%
backrefpage  = {\lowercase{s}ee p.}, % for single page number
backrefpages = {\lowercase{s}ee pp.} % for multiple page numbers
}
\addbibresource{/home/bvraghav/bibliography.bib}
\hypersetup{
colorlinks,
allcolors=.,
urlcolor=blue,
citecolor=gray,
linkcolor=red,
}
\usepackage{ragged2e}
\usepackage{tabularx}
\usepackage{enumitem}
\setlist{noitemsep}
\usepackage{parskip}
\usepackage[a4paper,margin=1.5in]{geometry}
\setcounter{secnumdepth}{2}
\author{Raghav B. Venkataramaiyer}
\date{Jan '26}
\title{Exercise\\\medskip
\large (The Geometric Model)}
\hypersetup{
 pdfauthor={Raghav B. Venkataramaiyer},
 pdftitle={Exercise},
 pdfkeywords={},
 pdfsubject={},
 pdfcreator={Emacs 30.2 (Org mode 9.7.11)}, 
 pdflang={English}}
\begin{document}

\maketitle
\setcounter{tocdepth}{2}
\tableofcontents

\section{Exercises: The Geometric Model (VR Geometry)}
\label{exercises-the-geometric-model-vr-geometry}
Based on the concepts of 3D modeling, transformations,
and orientations for Virtual Reality.
\subsection{Level 1: Easy (Foundational Concepts)}
\label{level-1-easy-foundational-concepts}
\begin{enumerate}
\item {\scriptsize{[}TH]}
Define a ``Geometric Model'' in the context of a
Virtual World Generator (VWG). \\
{\scriptsize{[}SOL]}
A mathematical description of the shape, size, and
position of objects in a 3D virtual space.

\item {\scriptsize{[}NM]}
A point \(P\) is located at \((2, 3, 5)\). If the
model is translated by a vector \(t = (1, -1, 2)\),
what are the new coordinates of \(P\)? \\
{\scriptsize{[}SOL]}
\((2+1, 3-1, 5+2) = (3, 2, 7)\).

\item {\scriptsize{[}SB]}
Explain why it is common to use triangles (meshes)
to represent complex 3D objects instead of
higher-order curved surfaces. \\
{\scriptsize{[}SOL]}
Triangles are computationally efficient, always
planar, and supported by GPU hardware.

\item {\scriptsize{[}TH]}
What are the three standard components of a 3D
transformation (often abbreviated as TRS)? \\
{\scriptsize{[}SOL]}
Translation, Rotation, and Scale.

\item {\scriptsize{[}NM]}
If a virtual world is defined in \(\mathbb{R}^3\),
what is the distance between point \(A(1, 0, 0)\)
and point \(B(4, 4, 0)\)? \\
{\scriptsize{[}SOL]}
\(\sqrt{(4-1)^2 + (4-0)^2 + (0-0)^2} = 5\).

\item {\scriptsize{[}TH]}
Define the difference between ``Global Coordinates''
and ``Local (Body) Coordinates.'' \\
{\scriptsize{[}SOL]}
Global: Fixed to world origin. Local: Fixed to the
object; moves with it.

\item {\scriptsize{[}SB]}
In a VR engine, you see an option to ``Parent'' a
sword to a character's hand. Based on geometric
modeling, what does this imply about their
coordinate systems? \\
{\scriptsize{[}SOL]}
The sword's transform is relative to the hand's
local coordinate system.

\item {\scriptsize{[}NM]}
Express a translation of 5 units along the x-axis
and -3 units along the z-axis as a 3D translation
vector. \\
{\scriptsize{[}SOL]}
\(t = (5, 0, -3)\).
\end{enumerate}
\subsection{Level 2: Medium (Transformations \& Rotations)}
\label{level-2-medium-transformations-rotations}
\begin{enumerate}
\setcounter{enumi}{8}
\item {\scriptsize{[}TH]}
Explain why 2D rotations can be represented by a
single angle \(\theta\), whereas 3D rotations
require more complex representations (like matrices
or quaternions). \\
{\scriptsize{[}SOL]}
2D has only one axis; 3D rotations are
non-commutative (order matters).

\item {\scriptsize{[}NM]}
Write the \(3 \times 3\) rotation matrix
\(R_z(\theta)\) for a rotation of \(90^\circ\)
around the Z-axis. \\
{\scriptsize{[}SOL]}
\([[0, -1, 0], [1, 0, 0], [0, 0, 1]]\).

\item {\scriptsize{[}SB]}
Discuss the phenomenon of ``Gimbal Lock.'' Which
rotation representation is most susceptible to it,
and why is it a problem for VR head tracking? \\
{\scriptsize{[}SOL]}
Euler Angles. Occurs when two axes align, losing a
degree of freedom.

\item {\scriptsize{[}NM]}
Apply a \(2 \times 2\) rotation matrix for \(\theta
    = 45^\circ\) to the 2D point \((1, 0)\). (Hint:
\(\cos 45^\circ = \sin 45^\circ =
    \frac{1}{\sqrt{2}}\)). \\
{\scriptsize{[}SOL]}
\((\frac{1}{\sqrt{2}}, \frac{1}{\sqrt{2}})\).

\item {\scriptsize{[}TH]}
What is a ``Homogeneous Transformation Matrix,'' and
what is the primary advantage of using a \(4 \times
    4\) matrix for 3D transformations? \\
{\scriptsize{[}SOL]}
A matrix that allows translation to be treated as a
multiplication, simplifying composition.

\item {\scriptsize{[}NM]}
Given a rotation matrix \(R\), prove that its
transpose \(R^T\) is equal to its inverse
\(R^{-1}\) (the Orthogonality property). \\
{\scriptsize{[}SOL]}
\(R \cdot R^T = I\) because rows/columns are
orthonormal.

\item {\scriptsize{[}SB]}
Compare the use of Euler Angles versus Quaternions
for interpolating between two camera
orientations. Why is one preferred for smooth
movement? \\
{\scriptsize{[}SOL]}
Quaternions allow for SLERP (Spherical Linear
Interpolation), which is smoother.

\item {\scriptsize{[}TH]}
Define ``Yaw,'' ``Pitch,'' and ``Roll'' in the context of
an aircraft or a VR headset. \\
{\scriptsize{[}SOL]}
Yaw: Y-axis; Pitch: X-axis; Roll: Z-axis.
\end{enumerate}
\subsection{Level 3: Hard (Advanced Modeling \& Kinematics)}
\label{level-3-hard-advanced-modeling-kinematics}
\begin{enumerate}
\setcounter{enumi}{16}
\item {\scriptsize{[}NM]}
A camera is located at \((0, 0, 0)\) looking toward
the negative Z-axis. An object is at \((0, 0,
    -10)\). If the camera moves to \((5, 0, 0)\) and
rotates \(90^\circ\) around the Y-axis (looking
toward the positive X-axis), what are the object's
coordinates in the \emph{camera's new local frame}? \\
{\scriptsize{[}SOL]}
The object is now at local \((0, 0, 5)\) relative
to the new camera orientation.

\item {\scriptsize{[}TH]}
Describe the ``Viewing Transformation.'' List the
steps required to move from a 3D world coordinate
to a 2D coordinate on the screen. \\
{\scriptsize{[}SOL]}
World \(\rightarrow\) Eye \(\rightarrow\) Canonical
View \(\rightarrow\) Screen.

\item {\scriptsize{[}NM]}
Compose a single \(4 \times 4\) homogeneous matrix
that first rotates an object \(30^\circ\) around
the X-axis and then translates it by \((0, 10,
    0)\). \\
{\scriptsize{[}SOL]}
\(M = T \times R\) (Translation applied after
rotation).

\item {\scriptsize{[}SB]}
If a VR system has high latency in updating the
``Geometric Model'' relative to the user's head
movement, describe the resulting visual artifacts
and their impact on the user. \\
{\scriptsize{[}SOL]}
``Judder'' or latency lag, leading to motion
sickness.

\item {\scriptsize{[}TH]}
Explain the concept of ``Double Covering'' in the
context of Unit Quaternions (\(q\) and \(-q\)
representing the same rotation). \\
{\scriptsize{[}SOL]}
Two points on the 4D hypersphere represent one 3D
rotation.

\item {\scriptsize{[}NM]}
Convert the quaternion \(q = ( \cos(\theta/2), 0,
    0, \sin(\theta/2) )\) back into its equivalent
rotation matrix form. \\
{\scriptsize{[}SOL]}
Result is the standard \(R_z(\theta)\) matrix.

\item {\scriptsize{[}SB]}
You are designing a VR ``Mirror'' where the user sees
their avatar. Explain the geometric transformation
required to reflect the avatar's movements across a
plane \(x = 0\). \\
{\scriptsize{[}SOL]}
Reflection matrix where \(x' = -x\).

\item {\scriptsize{[}NM]}
Calculate the result of rotating point \(v = (1, 0,
    0)\) by \(180^\circ\) around the Y-axis using the
quaternion formula \(v' = qvq^{-1}\). \\
{\scriptsize{[}SOL]}
\(v' = (-1, 0, 0)\).

\item {\scriptsize{[}TH]}
In the ``Rendering Pipeline,'' explain the role of
the ``Culling'' process. \\
{\scriptsize{[}SOL]}
Removing objects outside the Viewing Frustum to
optimize performance.
\end{enumerate}
\end{document}
