% Created 2026-01-14 Wed 13:27
% Intended LaTeX compiler: pdflatex
\documentclass[11pt]{article}
\usepackage[utf8x]{inputenc}
\usepackage[T1]{fontenc}
\usepackage{graphicx}
\usepackage{longtable}
\usepackage{wrapfig}
\usepackage{rotating}
\usepackage[normalem]{ulem}
\usepackage{amsmath}
\usepackage{amssymb}
\usepackage{capt-of}
\usepackage{hyperref}
\usepackage{minted}
\usepackage{parskip}
\graphicspath{ {./images/} }
\usepackage[labelformat=simple]{subcaption}
\renewcommand\thesubfigure{(\alph{subfigure})}
\usepackage[date=year,%
backend=biber,%
style=alphabetic,%
maxnames=5,%
minnames=3,%
maxalphanames=4,%
minalphanames=3,%
backref=true,%
doi=false,%
isbn=false,%
url=false,%
eprint=false]{biblatex}
\DefineBibliographyStrings{english}{%
backrefpage  = {\lowercase{s}ee p.}, % for single page number
backrefpages = {\lowercase{s}ee pp.} % for multiple page numbers
}
\addbibresource{/home/bvraghav/bibliography.bib}
\hypersetup{
colorlinks,
allcolors=.,
urlcolor=blue,
citecolor=gray,
linkcolor=red,
}
\usepackage{ragged2e}
\usepackage{tabularx}
\usepackage{enumitem}
\setlist{noitemsep}
\usepackage{parskip}
\usepackage[a4paper,margin=1.5in]{geometry}
\setcounter{secnumdepth}{2}
\author{Raghav B. Venkataramaiyer}
\date{Jan '26}
\title{Geometric Model}
\hypersetup{
 pdfauthor={Raghav B. Venkataramaiyer},
 pdftitle={Geometric Model},
 pdfkeywords={},
 pdfsubject={},
 pdfcreator={Emacs 30.2 (Org mode 9.7.11)}, 
 pdflang={English}}
\begin{document}

\maketitle
\setcounter{tocdepth}{2}
\tableofcontents

\section{Geometric Model}
\label{sec:orgb60aaec}

\subsection{Coordinates}
\label{sec:org5d3eb69}

\subsubsection*{Euclidean Plane}
\label{sec:org3fbe40c}
A point in \emph{(2-dimensional)} space \emph{(a.k.a. plane)}, is
defined as
\begin{align*}
  \mathbf{p}_i
  &= \begin{bmatrix} y\\x \end{bmatrix}_i
    = \begin{bmatrix} y_i\\x_i \end{bmatrix}
\end{align*}

The vector difference between two points is defined as,
\begin{align*}
  \mathbf{p}_i-\mathbf{p}_j
  &= \begin{bmatrix} y_i-y_j\\x_i-x_j \end{bmatrix}
\end{align*}

The distance between two points is a \emph{symmetric}
function of the two points, defined as,

\begin{align*}
  \delta_{ij}
  &= \|\mathbf{p}_i-\mathbf{p}_j\|_2
  \\
  \delta_{ij}^2 = \delta_{ji}^2
  &= \|\mathbf{p}_i-\mathbf{p}_j\|_2^2 =
    (y_i-y_j)^2+(x_i-x_j)^2 
\end{align*}
\subsubsection*{Euclidean Space}
\label{sec:org7544b45}

Extend the understanding to multiple dimensions, and we
get Euclidean Space,

A point in \emph{(d-dimensional)} space, is defined as
\begin{align*}
  \mathbf{p}
  &=  \begin{bmatrix}p_1\\\vdots\\p_d\end{bmatrix}
\end{align*}

The vector difference between two points is defined as,
\begin{align*}
  \mathbf{p}-\mathbf{q}
  &= \begin{bmatrix}
    p_1-q_1 \\ \vdots \\ p_d-q_d
  \end{bmatrix}
\end{align*}

The distance between two points is a \emph{symmetric}
function of the two points, defined as,

\begin{align*}
  \delta_{pq}
  &= \|\mathbf{p}-\mathbf{q}\|_2
  \\
  \delta_{pq}^2 = \delta_{qp}^2
  &= \|\mathbf{p}-\mathbf{q}\|_2^2 =
    (p_1-q_1)^2+\cdots+(p_d-q_d)^2 
\end{align*}
\subsubsection*{Polar Coordinates}
\label{sec:org7376889}
\(\mathbf{p} \equiv (r,\theta)\)

Defined by an origin, a pole, and each point in
\emph{(2-dimensional)} space is defined by
\begin{itemize}
\item \(r\), the distance between the point and origin; and
\item \(\theta\), the angle between pole and the line joining
the point and origin.
\end{itemize}
\subsubsection*{Spherical Coordinates}
\label{sec:org8946d25}
\(\mathbf{p} \equiv (r,\theta,\phi)\)

Extend the polar coordinates to three dimensional
space, we get spherical coordinates.
\subsubsection*{Geo-spherical Coordinates}
\label{sec:org3218c5a}
\(\mathbf{p} \equiv (\theta,\phi)\) \\
\textsc{or} \\
\(\mathbf{p} \equiv (1,\theta,\phi)\)

Useful when the space is a spherical surface, \emph{e.g.}
\begin{itemize}
\item The surface of the earth;
\item Projection surface \emph{(i.e. where the distance does not
matter.)}
\end{itemize}

Assume that for all points in space, \(r\) is constant;
and hence redundant; and thus a point is expressed only
in angular coordinates,
\subsection{Coordinate Systems and Frames of Reference}
\label{sec:org6f6c682}

Another way to define coordinate system is like
specifying a reference frame.  It's also referred to as
``space''.  \emph{E.g.}

\begin{itemize}
\item Earth revolves around the sun.

\begin{align*}
  \begin{bmatrix} y\\x \end{bmatrix}_t
  &= \begin{bmatrix} r\sin\omega t\\
    r\cos\omega t \end{bmatrix}
\end{align*}
where, \(\begin{bmatrix} y&x \end{bmatrix}_{t}^{\top}\)
are Earth's coordinates in ``Sun's space'' or in ``the
coordinate system with respect to the Sun.''

where, \(\omega\) is the \emph{(uniform)} angular velocity
of \emph{(centre of the)} earth; \(t\) is time; and that \(r\)
is a constant \emph{(the earth-to-sun distance)}.

\item The Earth rotates around its axis;

\begin{align*}
  \begin{bmatrix} y\\x \end{bmatrix}_t
  &= \begin{bmatrix} r\sin\omega t\\
    r\cos\omega t \end{bmatrix}
\end{align*}

where, \(\begin{bmatrix} y&x \end{bmatrix}_{t}^{\top}\)
are coordinates of Earth's surface in ``Earth Axis'
space'' or in ``the coordinate system with respect to
the Earth's Axis.''

where, \(\omega\) is the \emph{(uniform)} angular velocity
of earth's rotation about it's \emph{(diametric)} axis;
\(t\) is time; and that \(r\) is a constant \emph{(the
earth's radius)}.
\end{itemize}
\subsection{Linear Transformation}
\label{sec:org74a1af2}
From Linear Algebra, \\
with \(\mathbf{x}\in\mathbb{X}\) \\
\emph{implicitly read as vector \(\mathbf{x}\) in vector-space
\(\mathbb{X}\).}

Similarly \(\mathbf{y}\in\mathbb{Y}\), and \\
a linear map \(A:\mathbb{X}\to\mathbb{Y}\) \\
\emph{read as \(A\) maps \(\mathbb{X}\) to \(\mathbb{Y}\)} \\
And \(A\) is a matrix.

Please note that if \(\mathbb{X}\) is \(M\) dimensional and
\(\mathbb{Y}\) is \(N\) dimensional, then \(A\) is said to be
\(N\times M\) dimensional, so that,

\begin{align*}
\mathbf{y} &= A\mathbf{x}
\end{align*}
would imply that \(\mathbf{y}\) is a point in
\(\mathbb{Y}\) corresponding to \(\mathbf{x}\) in
\(\mathbb{X}\).

This operation is also known as linear transformation.
\subsection{Geometric Transformation}
\label{sec:org71ad93a}

\subsubsection*{Geometric Operations}
\label{sec:orgcfc330f}

\begin{enumerate}
\item Scale
\item Translate
\item Rotate
\item Shear
\item Project \emph{(or, Projection)}
\end{enumerate}
\subsubsection*{Scale}
\label{sec:org24b262a}

\begin{align*}
  \mathbf{y}
  &= A\mathbf{x} &= \begin{bmatrix}
    s_1x_1 \\
    s_2x_2
  \end{bmatrix} \\
  A &= \begin{bmatrix}
    s_1 & 0 \\
    0 & s_2
  \end{bmatrix}
\end{align*}
\subsubsection*{Rotate}
\label{sec:orgdd084d7}

\begin{align*}
  \mathbf{y}
  &= A\mathbf{x} &= \begin{bmatrix}
    x_1\cos\theta - x_2\sin\theta \\
    x_1\sin\theta + x_2\cos\theta
  \end{bmatrix} \\
  A &= \begin{bmatrix}
    \cos\theta & -\sin\theta \\
    \sin\theta & \cos\theta
  \end{bmatrix}
\end{align*}
\subsubsection*{Shear}
\label{sec:org6e099b2}

\begin{align*}
  \mathbf{y} &= A\mathbf{x} &= \begin{bmatrix}
    x_1+ax_2 \\ x_2
  \end{bmatrix} \\
  A &= \begin{bmatrix}
    1 & a \\ 0 & 1
  \end{bmatrix}
\end{align*}
\subsubsection*{Homogeneous Coordinates}
\label{sec:org4aa8d41}

\def\svgwidth{0.5\linewidth}
\input{images/RationalBezier2D.pdf_tex}

\(\mathbf{p}_{i}\) represents a ray in space rather than
a point.

\begin{align*}
  \mathbf{p}_i &= \begin{bmatrix}
    \lambda x \\ \lambda y \\ \lambda
  \end{bmatrix} = \begin{bmatrix}
    x \\ y \\ 1
  \end{bmatrix} && \forall \lambda\in\mathbb{R}
\end{align*}
\subsubsection*{Advantages of using homogeneous coordinates}
\label{sec:org23b18f4}

From linear algebra, we can represent,

\begin{align*}
  \mathbf{y} &= A\mathbf{x} \\
  \text{or, } y_i &= \sum_j a_{ij}x_j
\end{align*}

But what about the following case?
\begin{align*}
  \mathbf{y} &= \mathbf{x} + \mathbf{t}
\end{align*}

This can be represented as \(\mathbf{y} = A\mathbf{x}\),
if \(\mathbf{y}\) and \(\mathbf{x}\) are homogeneous
coordinates.

\emph{(Think about it)}
\subsubsection*{Translation}
\label{sec:orgb4b154e}
With homogeneous coordinates,

\begin{align*}
  \mathbf{y} &= A\mathbf{x} &= \begin{bmatrix}
    x_1 + t_1 \\
    x_2 + t_2 \\
    1
  \end{bmatrix} \\
  A &= \begin{bmatrix}
    1&0&t_1 \\ 0&1&t_1 \\ 0&0&1
  \end{bmatrix}
\end{align*}
\subsubsection*{Transformation Matrices With Homogeneous Coordinates}
\label{sec:org9ff2f4e}

\begin{itemize}
\item Scale
\label{sec:org3a5681f}

\begin{align*}
  A &= \begin{bmatrix}
    s_1 & 0 & 0 \\
    0 & s_2 & 0 \\
    0 & 0 & 1
  \end{bmatrix}
\end{align*}
\item Rotate
\label{sec:org3608444}

\begin{align*}
  A &= \begin{bmatrix}
    \cos\theta & -\sin\theta & 0 \\
    \sin\theta & \cos\theta & 0 \\
    0 & 0 & 1
  \end{bmatrix}
\end{align*}
\item Shear
\label{sec:orge6eb038}

\begin{align*}
  A &= \begin{bmatrix}
    1 & a & 0 \\ 0 & 1 & 0 \\ 0 & 0 & 1
  \end{bmatrix}
\end{align*}
\end{itemize}
\subsubsection*{Geometric Transformation in 3-or-more Dims}
\label{sec:org98609ec}

The matrices here represent transformation with
homogeneous coordinates; 
\begin{itemize}
\item Scale
\label{sec:org3a0f435}

\begin{align*}
  A &= \begin{bmatrix}
    s_1 & 0 & \cdots & 0 \\
    0 & s_2 & \cdots & 0 \\
    \vdots & \vdots & \ddots & \vdots \\
    0 & 0 & \cdots & 1
  \end{bmatrix}
\end{align*}
\item Translate
\label{sec:org35038e4}

\begin{align*}
  A &= \begin{bmatrix}
    1 & 0 & \cdots & t_1 \\
    0 & 1 & \cdots & t_2 \\
    \vdots & \vdots & \ddots & \vdots \\
    0 & 0 & \cdots & 1
  \end{bmatrix}
      = \begin{bmatrix}
        I & \mathbf{t} \\
        \boldsymbol{0}^T & 1
      \end{bmatrix}
\end{align*}
\item Shear
\label{sec:org7b86a71}

\begin{align*}
  A &= \begin{bmatrix}
    1 & a_{12} & a_{13} & \cdots & 0 \\
    0 & 1 & a_{23} & \cdots & 0 \\
    \vdots & \vdots & \vdots & \ddots & \vdots \\
    0 & 0 & 0 & \cdots & 1
  \end{bmatrix} = \begin{bmatrix}
    I+U_0 & \boldsymbol{0} \\
    \boldsymbol{0}^{\top} & 1
  \end{bmatrix}
\end{align*}

where, \(U_{0}\) is an upper triangular matrix with zero
diagonals.
\end{itemize}
\subsubsection*{Advanced Topics}
\label{sec:org8ab214e}

Projection and Rotation in 3D are considered to be
advanced topics in Computer Graphics.  The curious
readers are encouraged to follow the course UCS505 for
details.

As in practice, the students shall use mature
libraries, instead, to this effect.
\subsubsection*{Composing Transforms}
\label{sec:orge5df86e}

\begin{itemize}
\item Interpretation
\label{sec:org569957c}

A transformation is interpreted as to ``correspond from
one space to another.''
\item Example
\label{sec:orgb0bd557}

Let's consider our solar system.

\begin{enumerate}
\item A point that sits on the surface of the earth with
long-lat as \((\theta,\phi)\), is given as:

\begin{align*}
  \mathbf{p}_e &= \begin{bmatrix}
    r_e\cos\theta\cos\phi \\
    r_e\cos\theta\sin\phi \\
    r_e\sin\theta \\
    1
  \end{bmatrix}
\end{align*}

using homogeneous Cartesian coordinates.

\item The earth rotates around its axis, with a uniform
angular speed \(\omega_a\).  So in the axis-space, the
coordinates would be, loosely speaking,
\(\mathbf{p}_e\) rotated by \(\omega_{a} t\).  And
formally given as,
\begin{align*}
  \mathbf{p}_a &= A_{ea}\mathbf{p}_e \\
  A_{ae} &= \mathrm{rot}(\omega_a t)
\end{align*}

\item Similarly in sun-space, 
\begin{align*}
  \mathbf{p}_s &= A_{sa}\mathbf{p}_a =
                 A_{sa}A_{ae}\mathbf{p}_e \\
  A_{as} &= \mathrm{rot}(\omega_s t)
\end{align*}

\item Effectively, if \(\mathbf{p}_s =
   \mathbf{A}_{se}\mathbf{p}_e\), we have,

\begin{align*}
  A_{se} &= A_{sa}A_{ae}
\end{align*}

This method is called (de)composing.
\end{enumerate}
\end{itemize}
\subsection{Model-View-Projection}
\label{sec:org501cb20}

\begin{itemize}
\item \textbf{Model Space}
\label{sec:orgcf85fe7}
Continuing with the same example, let's assume that
``the sun'' is our ``model space.''  And \(A_{se}\) expresses
a point on the earth's surface in the sun-space.
\item \textbf{View Space}
\label{sec:org1896dc2}
We would like to view the system from a point in space
looking at the solar system so that the view is fully
captured within a \(60^{\circ}\) field of view.

If the point of view is located at \(\mathbf{p}_c\) in
\emph{model space}, and the camera is looking at point
\(\mathbf{p}_v\) in \emph{model space}; we should be able to
(de)compose the transformation from model-space to
``view space'' or ``camera space'' as a sequence of
geometric transformations.

Some practical references to the \texttt{glm::look\_at}
function:
\begin{enumerate}
\item \href{https://gamedev.stackexchange.com/a/189020}{GameDev Stack Exchange}
\item \href{https://www.geertarien.com/blog/2017/07/30/breakdown-of-the-lookAt-function-in-OpenGL/}{Geert Ariën's Blogpost}
\item \href{https://morning-flow.com/2023/02/06/the-math-behind-the-lookat-transform/}{Simón's Blogpost}
\end{enumerate}
\item \textbf{Projection Space}
\label{sec:orgc522dcc}
Depending upon whether the camera is orthogonal or
perspective, we finally capture the projection of ``the
scene,'' on an imaginary projection screen kept at
\(\begin{bmatrix}0&0&1\end{bmatrix}^{\top}\).  This is
called ``the projection space.''
\item \textbf{The MVP Transformation}
\label{sec:org79713bd}
To compute the coordinates of a point on object, say
\(\mathbf{p}_e\), as projected onto a projection space,
\(\mathbf{p}_{\pi}\), we compute this as,

\begin{align*}
  \mathbf{p}_{\pi} &= PVM\mathbf{p}_e
\end{align*}

where, \\
\(P\) is the transformation matrix from view space to
projection space; \\
\(V\) is the transformation matrix from model space to
view space; and \\
\(M\) is the transformation matrix from object space to
model space.
\end{itemize}
\section{Geometric Image Formation}
\label{sec:org0effca6}
\begin{figure}[htbp]
\centering
\includegraphics[width=.9\linewidth]{org-download-images/Image_Formation_and_Digital_Representation/2025-11-15_15-25-09_screenshot.png}
\caption{\label{fig:pinhole}\textbf{Pinhole Camera Model Diagram Extrinsic Intrinsic}: A diagram showing the geometry: the focal length, the image plane, and the projection of the 3D world point.  \emph{Image Courtesy}: \cite{GIMS18}}
\end{figure}


\textbf{The Pinhole Model} (Fig.~\ref{fig:pinhole})

The fundamental geometric model for any camera is the
\textbf{Pinhole Camera Model}.  It simplifies the complex
optics of a lens to a single aperture, or pinhole,
which projects a 3D point \((X, Y, Z)\) onto a 2D image
plane \((x, y)\).

Using homogeneous coordinates, this transformation can
be concisely expressed as a matrix multiplication:

$$s \mathbf{p} = \mathbf{K} [\mathbf{R} | \mathbf{t}]
\mathbf{P}$$

\begin{itemize}
\item \(\mathbf{P} \equiv \begin{bmatrix}X & Y & Z &
  1\end{bmatrix}^T\) is the 3D point in world
coordinates.
\item \(\mathbf{p} \equiv \begin{bmatrix} x & y & 1
  \end{bmatrix}^T\) is the homogeneous counterpart of 2D
point \((x, y)\) in image coordinates (projected to
3D).
\item \(s\) is an arbitrary scale factor (often \(=Z\), \emph{i.e.}
the depth).
\item \([\mathbf{R} | \mathbf{t}]\) is the \textbf{Extrinsic
Parameter Matrix} (a \(3 \times 4\) matrix combining a
\(3 \times 3\) rotation matrix \(\mathbf{R}\) and a \(3
  \times 1\) translation vector \(\mathbf{t}\)).  This
defines the camera's position and orientation
relative to the world coordinate system.
\item \(\mathbf{K}\) is the \textbf{Intrinsic Parameter Matrix} (a
\(3 \times 3\) upper-triangular matrix), which maps
normalized camera coordinates to pixel coordinates.
It depends on focal lengths (\(f_x, f_y\)) and the
principal point (\(c_x, c_y\)).
\end{itemize}

This model formalizes the loss of \emph{depth information}
and provides the framework for tasks like 3D
reconstruction (\textbf{Structure from Motion}) and position
estimation (\textbf{Visual Odometry}).
\section{Camera Tracking}
\label{sec:orgcfbdb4f}

\begin{itemize}
\item Calibration and registration are fundamental
processes in AR.
\item Calibration aligns the camera \emph{w.r.t.} the real
world.
\item Registration aligns digital content \emph{w.r.t.} the
camera.
\item Without proper calibration and registration, virtual
objects would appear to float, jitter, or be
incorrectly scaled, breaking the immersive illusion
of AR.
\end{itemize}
\subsection{Calibration}
\label{sec:org0dfff32}
Determine the camera properties,
\begin{itemize}
\item Intrinsic Params
\begin{itemize}
\item Focal length \((f_x, f_y)\)
\item Principal point \((c_x, c_y)\)
\item Skew coefficient : typically assumed to be zero.
\item Distortion coefficients \((k_1, k_2, p_1, p_2, k_3,
    \ldots)\) that account for radial and tangential
distortions.
\end{itemize}
\item Extrinsic Params (Camera Pose, \emph{i.e. its precise
position and orientation})
\begin{itemize}
\item Rotation matrix, \(R\)
\item Translation vector \(\mathbf{t}\)
\end{itemize}
\end{itemize}

(As covered in the geometric model)
\subsection{Registration}
\label{sec:org7885ecc}
Align virtual content with the real world in real-time,
\subsubsection*{Scene Information Extraction:}
\label{sec:org1f155da}
\begin{itemize}
\item Feature Detection
\item Sensor Fusion
\end{itemize}
\subsubsection*{Temporal Correspondence Estimation:}
\label{sec:org4a496f7}
\begin{itemize}
\item Correspondence across scene information
\item Marker-based AR: Detect markers and establish
correspondence
\item Markerless/SLAM-based: Use features to establish
correspondence.
\end{itemize}
\subsubsection*{Pose Estimation (Real-time tracking)}
\label{sec:org890aec0}
\begin{itemize}
\item Estimate 6 DoF camera pose using correspondences;
\item This is an iterative process of refining the pose
as new frames are captured, and new correspondences
are established.
\item Algorithms used:
\begin{itemize}
\item Perspective and point (PnP)
\item Bundle Adjustment
\item Filtering techniques (\emph{e.g.} Kalman filters,
particle filters)
\end{itemize}
\end{itemize}
\subsubsection*{Virtual Content Overlay}
\label{sec:orgbc118e9}
\begin{itemize}
\item Based on estimated camera pose, and predefined
transformation of virtual content relative to
\textbf{tracked} real world, virtual content is rendered
and projected onto the camera's view.
\item Three key operations, so that the virtual objects
appear to be a natural part of physical
environment:
\begin{itemize}
\item Estimating geometric transformation (namely
perspective, scale and rotation)
\item Estimating illumination model;
\item Occlusion handling.
\end{itemize}
\end{itemize}
\subsection{Registration Strategies}
\label{sec:orgb9600e9}
\begin{itemize}
\item Marker-based : Track the marker
\item Markerless : Track the natural features
\item Model-based : Track the geometric model of the real
world
\item Location-based : Based on GPS/IMU data, typically, in
an outdoor setting.
\end{itemize}
\subsection{Essentially}
\label{sec:orgb6c2bcf}
\begin{itemize}
\item Calibration is a one-time (or occasional) process to
characterize the camera itself; while
\item Registration is the continuous, real-time process, to
correctly place virtual objects in the augmented
view.
\end{itemize}
\section*{References}
\label{sec:orgdf5ea79}
\printbibliography[heading=none]
\end{document}
